\documentclass[a4,10pt,landscape]{article}
\usepackage{multicol}
\usepackage{calc}
\usepackage{ifthen}
\usepackage[landscape]{geometry}
\usepackage{amsmath,amsthm,amsfonts,amssymb}
\usepackage{color,graphicx,overpic}
\usepackage{hyperref}
\usepackage[defaultsans]{droidsans}
\usepackage[defaultmono]{droidmono}
\renewcommand*\familydefault{\sfdefault} %% Only if the base font of the document is to be typewriter style
\usepackage[T1]{fontenc}


\pdfinfo{
  /Title (coap-cheatsheet.pdf)
  /Creator (LaTeX)
  /Author (Markus Becker)
  /Subject (CoAP)
  /Keywords (CoAP)}

% This sets page margins to .5 inch if using letter paper, and to 1cm
% if using A4 paper. (This probably isn't strictly necessary.)
% If using another size paper, use default 1cm margins.
\ifthenelse{\lengthtest { \paperwidth = 11in}}
    { \geometry{top=.25in,left=.25in,right=.25in,bottom=.25in} }
    {\ifthenelse{ \lengthtest{ \paperwidth = 297mm}}
        {\geometry{top=.5cm,left=.5cm,right=.5cm,bottom=.5cm} }
        {\geometry{top=.5cm,left=.5cm,right=.5cm,bottom=.5cm} }
    }

% Turn off header and footer
\pagestyle{empty}

% Redefine section commands to use less space
\makeatletter
\renewcommand{\section}{\@startsection{section}{1}{0mm}%
                                {-1ex plus -.5ex minus -.2ex}%
                                {0.5ex plus .2ex}%x
                                {\normalfont\large\bfseries}}
\renewcommand{\subsection}{\@startsection{subsection}{2}{0mm}%
                                {-1ex plus -.5ex minus -.2ex}%
                                {0.5ex plus .2ex}%
                                {\normalfont\normalsize\bfseries}}
\renewcommand{\subsubsection}{\@startsection{subsubsection}{3}{0mm}%
                                {-1ex plus -.5ex minus -.2ex}%
                                {1ex plus .2ex}%
                                {\normalfont\small\bfseries}}
\makeatother

% Define BibTeX command
\def\BibTeX{{\rm B\kern-.05em{\sc i\kern-.025em b}\kern-.08em
    T\kern-.1667em\lower.7ex\hbox{E}\kern-.125emX}}

% Don't print section numbers
\setcounter{secnumdepth}{0}


\setlength{\parindent}{0pt}
\setlength{\parskip}{0pt plus 0.5ex}

%My Environments
\newtheorem{example}[section]{Example}
% -----------------------------------------------------------------------

\begin{document}
\raggedright
\footnotesize
\begin{multicols}{3}


% multicol parameters
% These lengths are set only within the two main columns
%\setlength{\columnseprule}{0.25pt}
\setlength{\premulticols}{1pt}
\setlength{\postmulticols}{1pt}
\setlength{\multicolsep}{1pt}
\setlength{\columnsep}{1pt}

\begin{center}
     \Large{\underline{Constrained Application Protocol}} \\
     {\tiny
       (RFC 6690, RFC 7252, RFC 7959, RFC 7641)
     }
\end{center}

The Constrained Application Protocol (CoAP) is a specialized web
transfer protocol for use with constrained nodes and constrained
(e.g., low-power, lossy) networks.

\section{CoAP Message Format}
{\tiny
\begin{verbatim}
 0                   1                   2                   3
 0 1 2 3 4 5 6 7 8 9 0 1 2 3 4 5 6 7 8 9 0 1 2 3 4 5 6 7 8 9 0 1
+-+-+-+-+-+-+-+-+-+-+-+-+-+-+-+-+-+-+-+-+-+-+-+-+-+-+-+-+-+-+-+-+
|Ver| T |  TKL  |      Code     |          Message ID           |
+-+-+-+-+-+-+-+-+-+-+-+-+-+-+-+-+-+-+-+-+-+-+-+-+-+-+-+-+-+-+-+-+
|   Token (if any, TKL bytes) ...
+-+-+-+-+-+-+-+-+-+-+-+-+-+-+-+-+-+-+-+-+-+-+-+-+-+-+-+-+-+-+-+-+
|   Options (if any) ...
+-+-+-+-+-+-+-+-+-+-+-+-+-+-+-+-+-+-+-+-+-+-+-+-+-+-+-+-+-+-+-+-+
|1 1 1 1 1 1 1 1|    Payload (if any) ...
+-+-+-+-+-+-+-+-+-+-+-+-+-+-+-+-+-+-+-+-+-+-+-+-+-+-+-+-+-+-+-+-+
\end{verbatim}
Ver: Version, T: Type, TKL: Token Length
}

\section{Message types}
{\tiny
\begin{verbatim}
+------+-----------------+
| Type | Name            |
+------+-----------------+
|    0 | CONfirmable     |
|    1 | NON-confirmable |
|    2 | ACKnowledgement |
|    3 | ReSeT           |
+------+-----------------+
\end{verbatim}
}

\section{Method codes}
{\tiny
\begin{verbatim}
+------+--------+
| Code | Name   |
+------+--------+
| 0.00 | EMPTY  |
+------+--------+
| 0.01 | GET    |
| 0.02 | POST   |
| 0.03 | PUT    |
| 0.04 | DELETE |
+------+--------+
\end{verbatim}
}

\section{Response codes}

{\tiny
\begin{verbatim}
 0
 0 1 2 3 4 5 6 7
+-+-+-+-+-+-+-+-+
|class|  detail |
+-+-+-+-+-+-+-+-+
\end{verbatim}
}

{\tiny
\begin{verbatim}
+------------------+------------------------------+--------------+
| Code             | Description                  |              |
+------------------+------------------------------+--------------+
| 2.01 (65, 0x41)  | Created                      | Success      |
| 2.02 (66, 0x42)  | Deleted                      |              |
| 2.03 (67, 0x43)  | Valid                        |              |
| 2.04 (68, 0x44)  | Changed                      |              |
| 2.05 (69, 0x45)  | Content                      |              |
| 2.31 (95, 0x5F)  | Continue                     |              |
+------------------+------------------------------+--------------+
| 4.00 (128, 0x80) | Bad Request                  | Client Error |
| 4.01 (129, 0x81) | Unauthorized                 |              |
| 4.02 (130, 0x82) | Bad Option                   |              |
| 4.03 (131, 0x83) | Forbidden                    |              |
| 4.04 (132, 0x84) | Not Found                    |              |
| 4.05 (133, 0x85) | Method Not Allowed           |              |
| 4.06 (134, 0x86) | Not Acceptable               |              |
| 4.08 (136, 0x88) | Request Entity Incomplete    |              |
| 4.12 (140, 0x8C) | Precondition Failed          |              |
| 4.13 (141, 0x8D) | Request Entity Too Large     |              |
| 4.15 (143, 0x8F) | Unsupported Content-Format   |              |
+------------------+------------------------------+--------------+
| 5.00 (160, 0xA0) | Internal Server Error        | Server Error |
| 5.01 (161, 0xA1) | Not Implemented              |              |
| 5.02 (162, 0xA2) | Bad Gateway                  |              |
| 5.03 (163, 0xA3) | Service Unavailable          |              |
| 5.04 (164, 0xA4) | Gateway Timeout              |              |
| 5.05 (165, 0xA5) | Proxying Not Supported       |              |
+------------------+------------------------------+--------------+
\end{verbatim}
}

\section{Options}

{\tiny
\begin{verbatim}
  0   1   2   3   4   5   6   7
+---------------+---------------+
|  Option Delta | Option Length |   1 byte
+---------------+---------------+
/         Option Delta          /   0-2 bytes
\          (extended)           \
+-------------------------------+
/         Option Length         /   0-2 bytes
\          (extended)           \
+-------------------------------+
/         Option Value          /   0 or more bytes
+-------------------------------+
\end{verbatim}
}

{\tiny
\begin{verbatim}
+-----+----+---+---+---+----------------+--------+--------+-------------+
| No. | C  | U | N | R | Name           | Format | Length | Default     |
+-----+----+---+---+---+----------------+--------+--------+-------------+
|   1 | x  |   |   | x | If-Match       | opaque | 0-8    | (none)      |
|   3 | x  | x | - |   | Uri-Host       | string | 1-255  | (see note 1)|
|   4 |    |   |   | x | ETag           | opaque | 1-8    | (none)      |
|   5 | x  |   |   |   | If-None-Match  | empty  | 0      | (none)      |
|   7 | x  | x | - |   | Uri-Port       | uint   | 0-2    | (see note 1)|
|   8 |    |   |   | x | Location-Path  | string | 0-255  | (none)      |
|  11 | x  | x | - | x | Uri-Path       | string | 0-255  | (none)      |
|  12 |    |   |   |   | Content-Format | uint   | 0-2    | (none)      |
|  14 |    | x | - |   | Max-Age        | uint   | 0-4    | 60          |
|  15 | x  | x | - | x | Uri-Query      | string | 0-255  | (none)      |
|  17 | x  |   |   |   | Accept         | uint   | 0-2    | (none)      |
|  20 |    |   |   | x | Location-Query | string | 0-255  | (none)      |
|  28 |    |   | x |   | Size2          | uint   | 0-4    | (none)      |
|  35 | x  | x | - |   | Proxy-Uri      | string | 1-1034 | (none)      |
|  39 | x  | x | - |   | Proxy-Scheme   | string | 1-255  | (none)      |
|  60 |    |   | x |   | Size1          | uint   | 0-4    | (none)      |
+-----+----+---+---+---+----------------+--------+--------+-------------+
\end{verbatim}
C=Critical, U=Unsafe, N=No-Cache-Key, R=Repeatable \\
Note 1: taken from destination address/port of request message
}

\section{Content-Formats}
{\tiny
\begin{verbatim}
+--------------------------+-----+
| Media type               | Id. |
+--------------------------+-----+
| text/plain;charset=utf-8 |   0 |
| application/link-format  |  40 |
| application/xml          |  41 |
| application/octet-stream |  42 |
| application/exi          |  47 |
| application/json         |  50 |
| application/cbor         |  60 |
+--------------------------+-----+
\end{verbatim}
}

\section{URI schemes}
{\tiny
\begin{verbatim}
coap-URI = "coap:" "//" host [ ":" port ] path-abempty [ "?" query ]
coaps-URI = "coaps:" "//" host [ ":" port ] path-abempty [ "?" query ]
\end{verbatim}
}

\section{Transmission parameters}
{\tiny
\begin{verbatim}
+-------------------+---------------+
| name              | default value |
+-------------------+---------------+
| ACK_TIMEOUT       | 2 seconds     |
| ACK_RANDOM_FACTOR | 1.5           |
| MAX_RETRANSMIT    | 4             |
| NSTART            | 1             |
| DEFAULT_LEISURE   | 5 seconds     |
| PROBING_RATE      | 1 Byte/second |
+-------------------+---------------+
\end{verbatim}
}

\section{Link Format .well-known/core}

Link format can be used to describe hosted resources, their
attributes, and other relationships between links.

Example:
{\tiny
\begin{verbatim}
REQ: GET /.well-known/core

RES: 2.05 Content

</sensors>;ct=40;title="Sensor Index",
</sensors/temp>;rt="temperature-c";if="sensor",
</sensors/light>;rt="light-lux";if="sensor",
<http://www.example.com/sensors/t123>;anchor="/sensors/temp";rel="describedby",
</t>;anchor="/sensors/temp";rel="alternate"
\end{verbatim}
}

ABNF:
{\tiny
\begin{verbatim}
Link            = link-value-list
link-value-list = [ link-value *[ "," link-value ]]
link-value     = "<" URI-Reference ">" *( ";" link-param )
link-param     = ( ( "rel" "=" relation-types )
               / ( "anchor" "=" DQUOTE URI-Reference DQUOTE )
               / ( "rev" "=" relation-types )
               / ( "hreflang" "=" Language-Tag )
               / ( "media" "=" ( MediaDesc
                      / ( DQUOTE MediaDesc DQUOTE ) ) )
               / ( "title" "=" quoted-string )
               / ( "title*" "=" ext-value )
               / ( "type" "=" ( media-type / quoted-mt ) )
               / ( "rt" "=" relation-types )
               / ( "if" "=" relation-types )
               / ( "sz" "=" cardinal )
               / ( link-extension ) )
link-extension = ( parmname [ "=" ( ptoken / quoted-string ) ] )
               / ( ext-name-star "=" ext-value )
ext-name-star  = parmname "*" ; reserved for RFC-2231-profiled
                              ; extensions.  Whitespace NOT
                              ; allowed in between.
ptoken         = 1*ptokenchar
ptokenchar     = "!" / "#" / "$" / "%" / "&" / "'" / "("
               / ")" / "*" / "+" / "-" / "." / "/" / DIGIT
               / ":" / "<" / "=" / ">" / "?" / "@" / ALPHA
               / "[" / "]" / "^" / "_" / "`" / "{" / "|"
               / "}" / "~"
media-type     = type-name "/" subtype-name
quoted-mt      = DQUOTE media-type DQUOTE
relation-types = relation-type
               / DQUOTE relation-type *( 1*SP relation-type ) DQUOTE
relation-type  = reg-rel-type / ext-rel-type
reg-rel-type   = LOALPHA *( LOALPHA / DIGIT / "." / "-" )
ext-rel-type   = URI
cardinal       = "0" / ( %x31-39 *DIGIT )
LOALPHA        = %x61-7A   ; a-z
quoted-string  = <defined in [RFC2616]>
URI            = <defined in [RFC3986]>
URI-Reference  = <defined in [RFC3986]>
type-name      = <defined in [RFC4288]>
subtype-name   = <defined in [RFC4288]>
MediaDesc      = <defined in [W3C.HTML.4.01]>
Language-Tag   = <defined in [RFC5646]>
ext-value      = <defined in [RFC5987]>
parmname       = <defined in [RFC5987]>
\end{verbatim}
}

\section{Block}

In order to transfer larger payloads with CoAP $-$ for instance, for
firmware updates $-$ the Block option can be used.

{\tiny
\begin{verbatim}
+-----+---+---+---+---+----------------+--------+--------+----------+
| No. | C | U | N | R | Name           | Format | Length | Default  |
+-----+---+---+---+---+----------------+--------+--------+----------+
|  23 | x | x | - | - | Block2         | uint   | 0-3 B  | (none)   |
|  27 | x | x | - | - | Block1         | uint   | 0-3 B  | (none)   |
+-----+---+---+---+---+----------------+--------+--------+----------+
\end{verbatim}
}
{\tiny
\begin{verbatim}
 0
 0 1 2 3 4 5 6 7
+-+-+-+-+-+-+-+-+
|  NUM  |M| SZX |
+-+-+-+-+-+-+-+-+

 0                   1
 0 1 2 3 4 5 6 7 8 9 0 1 2 3 4 5
+-+-+-+-+-+-+-+-+-+-+-+-+-+-+-+-+
|          NUM          |M| SZX |
+-+-+-+-+-+-+-+-+-+-+-+-+-+-+-+-+

 0                   1                   2
 0 1 2 3 4 5 6 7 8 9 0 1 2 3 4 5 6 7 8 9 0 1 2 3
+-+-+-+-+-+-+-+-+-+-+-+-+-+-+-+-+-+-+-+-+-+-+-+-+
|                   NUM                 |M| SZX |
+-+-+-+-+-+-+-+-+-+-+-+-+-+-+-+-+-+-+-+-+-+-+-+-+
\end{verbatim}
}

\section{Observe}

In order to follow state changes of CoAP resources the Observe option
can be used.

{\tiny
\begin{verbatim}
+-----+---+---+---+---+---------+------------+-----------+---------+
| No. | C | U | N | R | Name    | Format     | Length    | Default |
+-----+---+---+---+---+---------+------------+-----------+---------+
|   6 |   | x | - |   | Observe | uint       | 0-3 B     | (none)  |
+-----+---+---+---+---+---------+------------+-----------+---------+
\end{verbatim}
}

\section{References}
This cheatsheet is based on and heavily stole from the following
documents:\\

{\tiny
Link-format: \url{http://tools.ietf.org/html/rfc6690}\\
CoAP: \url{http://tools.ietf.org/html/rfc7252}\\
Block: \url{http://tools.ietf.org/html/rfc7959}\\
Observe: \url{http://tools.ietf.org/html/rfc7641}\\
}
% You can even have references
%\rule{0.3\linewidth}{0.25pt}
%\scriptsize
%\bibliographystyle{abstract}
%\bibliography{refFile}
\end{multicols}
\end{document}
